\subsection{Consideraciones Generales}

Cada video est\'a compuesto por frames de $h$ p\'ixeles de altura y $w$ p\'ixeles de
ancho, donde cada pixel es un valor entero del $0$ al $255$, inclusive. A la
vez, se consideran $c_0$ de estos frames en sucesi\'on para formar el video.

Para alentar el video nosotros tomamos un factor $q > 1$ (equivalente al
argumento $c + 1$), y generamos un video con $c_1 = (c_0 - 1) \cdot q + 1$ frames. A cada frame
le aplicamos cierto \textit{m\'etodo}, dependiendo de nuestras restricciones,
para aproximar los frames restantes entre el video original y el video alentado.

Cada uno de los m\'etodos que usamos opera con un solo pixel y el cambio de su
valor a trav\'es del tiempo, as\'i que no es necesario que conozca todo un frame
entero. Por esto, tomando el dominio $\mathlarger{\Pi} = \mathbb{N} \cap \left[ 0, 255
\right]$ como el valor de los pixeles en el frame o imagen, definimos cada m\'etodo
como una funci\'on que toma dos par\'ametros:

\[
\begin{split}
P \in \mathlarger{\Pi^{c_0}} & : \text{un frame (o vector de p\'ixeles)} \\
c_1 \in \mathbb{N} & : \text{la cantidad de frames en la pel\'icula luego de ser procesada con un m\'etodo}
\end{split}
\]

Y a la vez devuelve un par\'ametro del tipo

\[
A \in \mathlarger{\Pi^{c_1}} : \text{el frame $P$ en la pel\'icula final}
\]

El enunciado tambi\'en especifica el par\'emtro $f$, el framerate del video,
pero adem\'as de hacerle una transformaci\'on simple al video final no influye a
ninguno de nuestros m\'etodos.

Durante el resto de la resoluci\'on usamos t\'erminos como ``cercano'' y
``lejano''. Se debe entender que, como los m\'etodos trabajan sobre un solo
pixel y como cambia en el tiempo, estos se refieren a t\'erminos temporales y
no espaciales.

\subsection{Vecino M\'as Cercano}

El m\'etodo del vecino m\'as cercano implica, por cada pixel en el frame
resultante, darle el valor del pixel m\'as cercano \footnote{temporalmente} en
el frame original. Para lograr esto, podemos definirlo como la siguiente
sucesi\'on:

\[
\begin{split}
q & = (c_1 - 1) / (c_0 - 1) \\
A_i & = P_{\left[i / q \right]} \hspace{.2\textwidth} 1 \leq i \leq h*w
\end{split}
\]

Donde $\left[x\right]$ representa el n\'umero entero m\'as cercano a $x$.

\subsection{Interpolaci\'on Lineal}

El m\'etodo de la interpolaci\'on lineal consiste en crear una funci\'on lineal
entre cada par de pixeles consecutivos \footnote{de nuevo, en el sentido
temporal} de $P$, y estimar los valores del vector $A$ desde el valor que toma
cada funci\'on en los puntos intermedios.

La funci\'on lineal entre dos puntos $\left(x_0, y_0\right)$ y $\left(x_1,
y_1\right)$ debe satisfacer la f\'ormula

\[
\frac{y - y_0}{x - x_0} = \frac{y_1 - y_0}{x_1 - x_0}
\]

Por lo tanto, podemos definir la funci\'on $f$ que represente la interpolaci\'on
lineal entre esos dos puntos como

\[
f(x) = y_0 + (y_1 - y_0) \frac{x - x_0}{x_1 - x_0}
\]

Como debemos definir esta f\'ormula para las interpolaciones entre varios puntos en
el plano discreto, definimos el m\'etodo como la sucesi\'on

\[
scale = (c_1 - 1) / (c_0 - 1)
\]

\[
\begin{split}
(\forall i & < c_1) \\
q & = \left\lfloor \frac{i}{scale} \right\rfloor \\ 
z & = \left\lceil \frac{i}{scale} \right\rceil \\
{x_0}^{(i)} & = q * scale \\
{y_0}^{(i)} & = P_{{q}^{(i)}} \\
{x_1}^{(i)} & = z * scale \\
{y_1}^{(i)} & = P_{{z}^{(i)}} \\
A_i & = {y_0}^{(i)} + ({y_0}^{(i)} - {y_1}^{(i)}) \cdot \frac{i - {x_0}^{(i)}}{{x_0}^{(i)} - {x_1}^{(i)}}
\end{split}
\]

\subsection{Interpolaci\'on por Splines}

En este m\'etodo definimos un spline c\'ubico n\'atural $S$ que interpole todos
los valores en $P$ para lograr conseguir los valores intermedios de $A$.
Nosotros definimos este spline como una funci\'on ``flu\'ida'' que pase por
todos los puntos $(x_0, y_0)$ definidos en $P$. Para lograr esto, separamos $S$
en diferentes subfunciones, $S_{0 \ldots c_0 - 1}$ tal que cada partici\'on sea
un polinomio c\'ubico definido entre dos puntos de $P$, y la derivada primera y
segunda de cada par consecutivos sean iguales. En otras palabras,

\[
\begin{split}
S(x) & =
\begin{cases}
	S_0(x) & \text{si } x \in [x_0, x_1] \\
	\vdots \\
	S_{c_1 - 1} & \text{si } x \in [x_{n - 1}, x_n]
\end{cases} \\
S_i(x) & = a_i + b_i(x - x_i) + c_i(x - x_i)^2 + d_i(x - x_i)^3 \\
S_i(x_i + 1) & = S_{x + 1}(x_i + 1) \; \forall i \in [0, n - 2] \\
S'_i(x_i + 1) & = S'_{x + 1}(x_i + 1) \; \forall i \in [0, n - 2] \\
S''_i(x_i + 1) & = S''_{x + 1}(x_i + 1) \; \forall i \in [0, n - 2] \\
S''(x_0) & = S''(x_n) = 0
\end{split}
\]

Como en cada $x_i$ $S(x_i) = S_i(x_i)$, y $S_i(x_i) = a_i$, podemos ver que

\[
a_i = y_i
\]

Tambi\'en, ya que $S_i(x_i + 1) = y_{i + 1}$,

\[
a_i + b_i(x_{i + 1} - x_i) + c_i(x_{i + 1} - x_i)^2 + d_i(x_{i + 1} - x_i)^3 = a_{i + 1}
\]

Derivando cualquier polinomio c\'ubico $S_i$ obtenemos que

\[
\begin{split}
S_i'(x) & = b_i + 2 c_i (x - x_i) + 3 d_i (x - x_i)^2 \\
S_i''(x) & = 2 c_i + 6 d_i (x - x_i)
\end{split}
\]

Por lo tanto, ya que las derivadas primera y segunda de los polinomios
consecutivos deben ser iguales,

\[
b_i + 2 c_i (x_{i + 1} - x_i) + 3 d_i (x_{i + 1} - x_i)^2 = b_{i + 1}
\]
\[
2 c_i + 6 d_i (x_{i + 1} - x_i) = 2 c_{i + 1}
\]

Y, como el spline es n\'atural y la derivada segunda en el primer y \'ultimo
punto debe ser $0$,

\[
2 c_0 = 0
\]
\[
2 c_{n - 1} + 6 d_{n - 1} (x_n - x_{n - 1}) = 0
\]

Resumiendo, y sabiendo que $x_{i + 1} - x_i = q \forall i \in [0, n - 2]$
tenemos las siguientes ecuaciones i para los factores de los polinomios:

\begin{enumerate}
\item $ a_i = y_i $
\item $ a_i + b_i q + c_i q^2 + d_i q^3 $
\item $ b_i + 2 c_i q + 3 d_i h_i^2 = b_{i + 1} $
\item $ 2 c_i + 6 d_i q = 2 c_{i + 1} $
\item $ c_0 = 0 $
\end{enumerate}

Despejando la ecuaci\'on 4, obtenemos

\[
d_i = \frac{1}{3 q} c_{i + 1} - c_i
\]

Y sustituyendo este $d$ y el $a$ de la ecuaci\'on 1 en la ecuaci\'on 2 obtenemos
que

\[
\begin{split}
b_i & = \frac{1}{q} (y_{i + 1} - y_i - c_i q^2 - d_i q^3) \\
& = \frac{1}{q} (y_{i + 1} - y_i - c_i q^2 - \frac{1}{q} (c_{i + 1} - c_i) q^3 \\
& = \frac{1}{q} (y_{i + 1} - y_i) - c_i q \frac{1}{3} (c_{i + 1} - c_i) q \\
& = \frac{1}{q} * (y_{i + 1} - y_i) - \frac{q}{3} (2 c_i + c_{i + 1})
\end{split}
\]

Substituyendo el valor de $b_i$ en la ecuaci\'on 3,

\[
\begin{split}
0 & = b_i - b_{i + 1} + 2 c_i q + 3 d_i q^2 \\
& = \frac{1}{q} (y_{i + 1} - y_i) - \frac{q}{3} (2 c_i + c_{i + 1}) - \frac{1}{q} (y_{i + 2) - y_{i + 1}} + \frac{q}{3} (2 c_{i + 1} + c{i + 2}) + 2 c_i q + q (c_{i + 1} - c_i) \\
& = \left[ \frac{1}{q} (y_{i + 1} - y_i) - \frac{1}{q} (y_{i + 2} - y_{i + 1}) \right] + \frac{2}{3} q c_i - \frac{1}{3} q c_{i + 1} + \frac{2}{3} q c_{i + 1} + \frac{1}{3} q c_{i + 2} + 2 q c_i + q c_{i + 1} - q c_i \\
& = \frac{1}{q} \left[ y_{i + 2} - 2 y_{i + 1} + y_i \right] + \frac{1}{3} q c_i + \frac{4}{3} q c_{i + 1} + \frac{1}{3} q c_{i + 2}
\end{split}
\]

Equivalentemente,

\[
q c_i + 4 q c_{i + 1} + q c_{i + 2} = \frac{3}{q} \left[ y_{i + 2} - 2 y_{i + 1} + y_i \right]
\]

Esto es un sistema de $n - 1$ ecuaciones con $n + 1$ incognitas ($c_0 \cdots
c_n$). Tomando $c_0 = 0$ y $c_n = 0$ tenemos un sistema de $n + 1$ ecuaciones
que se puede expresar matricialmente de la forma

\[
\begin{bmatrix}
1 & 0 & 0 & 0 \\
q & 4 q & 4 q & 0 \\
0 & q & 4 q & q \\
&&&& \ddots \\
&&&&& q & 4 q & q \\
&&&&& 0 & 0 & 1
\end{bmatrix}
\begin{bmatrix}
c_0 \\
c_1 \\
c_2 \\
\vdots \\
c_{n - 1} \\
c_n
\end{bmatrix}
=
\frac{3}{q} \cdot \begin{bmatrix}
0 \\
y_0 - 2 y_1 + y_2 \\
y_1 - 2 y_2 + y_3 \\
\vdots \\
y_{n - 2} - 2 y_{n - 1} + y_n \\
0
\end{bmatrix}
\]

$A$ es una matriz banda de ancho 1, ya que todos sus valores fuera de la
diagonal principal y las dos diagonales son iguales a $0$, y como es
estrictamente diagonal dominante es inversible; entonces hay una soluci\'on
\'unica al sistema.

M\'as a\'un, como $A$ es tridiagonal se puede guardar en una matriz esparsa que
contenga solo las diagonales y su complejidad espacial sea $\mathbb(O)(n)$ de
espacio. Tambi\'en, el sistema puede ser resuelto haciendo eliminaci\'on
Gaussiana sobre la matriz en complejidad temporal $\mathbb(O)(n)$.

Finalmente, haciendo resuelto este sistema de ecuaciones y consiguiendo los
$c_i$ de cada polinomio dentro del spline, podemos volver a las ecuaciones
anteriores y definir los otros parametros como

\[
\begin{split}
a_i & = y_i \\
b_i & = \frac{1}{q} (y_{i + 1} - y_i) - \frac{1}{3} q (2 c_i + c_{i + 1}) \\
d_i & = \frac{1}{3 q} (c_{i + 1} - c_i)
\end{split}
\]

Para ayudar al an\'alisis, usamos un parametro extra para este m\'etodo que es el 
\textbf{tama\~no de bloque} y lo vamos a definir como
$reset$ \'o $r$ (que indica que los splines se definen entre $r$ puntos). Eso es,
$S_0 \ldots S_{r - 1}$ forman un spline, $S_r \ldots S_{2r - 1}$ forman otro,
etc. Por simplicidad, en el caso que el \'ultimo spline contenga menos de $r$
puntos lo unimos con el anterior.

De esta manera, y como ya tenemos definidos todos los $S_i$, podemos definir $A$ como

\[
\begin{split}
j & = \left\lfloor \frac{i}{q} \right\rfloor \\
A_i & = S_j(i - j \cdot q)
\end{split}
\]
