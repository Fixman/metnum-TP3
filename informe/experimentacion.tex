En esta secci\'on evaluaremos los 3 m\'etodos implementados de manera de poder comparar su velocidad de corrida (complejidad), determinar la calidad de regeneraci\'on de videos, artifacts producidos, etc. . Para eso, no solo usaremos los videos provistos por la c\'atedra, sino que adem\'as hemos incluido otros 4 videos que, si bien no todos aparecer\'an en los gr\'aficos de resultados, nos servir\'an para sacar conclusiones sobre los comportamientos de los m\'etodos a partir de las caracter\'isticas del video. Estos 4 videos son los siguientes:

\begin{itemize}
\item \textbf{slowmovesscene.mp4}: Es una escena del conocido video BigBuckBunny que tiene la caracter\'itica de ser un video de movimientos lentos y suaves, con casi ning\'un cambio de c\'amara. Esto lleva a que la variaci\'on entre frames sea ligera. 
\item \textbf{fastmovesscene.mp4}: Este video es una escena de una pelea con movimientos de alta velocidad, lo cual significa que las diferencias entre frame y frame son notorias.
\item \textbf{nocamerachanges.mp4}: Este video no posee ning\'un cambio de c\'amara, y los movimientos dentro del mismo son moderados, lo cual impide que existan altos contrastes entre frames.
\item \textbf{camerachanges.mp4}: Es un video que posee grandes cantidades de cambios de c\'amara que originan fuertes contrastes entre frames continuos.
\end{itemize}

El porque de la selecci\'on de estos videos es la siguiente: dado que todos los m\'etodos generan frames entre cada par de frames del video de entrada, buscamos analizar como reaccionan los mismos cuando el par de frames que utilizan para generar los nuevos frames son, o muy parecidos entre s\'i (escenas lentas o de movimientos suaves), o notoriamento diferentes (cambios de c\'amara bruscos o escenas de alta velocidad). Adem\'as, creemos que los videos con cambios de c\'amara o de alta velocidad son m\'as propensos a producir los llamados ``artifacts'' cuando se alentan en gran medidad. 

A continuaci\'on, analizaremos como los 3 m\'etodos implementados var\'ian en tiempo y el error (\textbf{ECM} y \textbf{PSNR}) generado entre frames reconstruidos con estos m\'etodos contra los frames del video original.