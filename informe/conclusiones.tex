Descubrimos que la generaci\'on artificial de frames que nos ofrecen los m\'etodos implementados para producir un efecto de c\'amara lenta es muy \'util cuando se desea simular este efecto sin que la transmici\'on de video sea nativamente lenta. 

Entre los 3 m\'etodos, el del m\'as cercano, si bien es el m\'as r\'apido, produce un efecto no deseado al alentar los videos en gran medida con el mismo, generando algo m\'as parecido a un slide de im\'agenes que a un video. Es por eso que concluimos que, m\'as all\'a de su velocidad, no es una opci\'on viable para generar el efecto de c\'amara lenta que uno esperar\'ia. Con respecto a interpolaci\'on lineal y splines, las velocidades de ejecuci\'on de ambos var\'ian en gran medida, siendo el m\'etodo de splines el m\'as lento en correr. Sin embargo, a pesar de que ambos generan ``ghosting'', el video resultante de splines tiende a ser m\'as fluido, ya que la transici\'on generada entre los nuevos frames y los originales es m\'as cercana a lo esperado. Es por esto, que al momento de decidir cual es mejor, todo depender\'a del poder de c\'omputo con el que se cuenta, dado que generar el efecto con splines es m\'as costoso. La elecci\'on depender\'a entonces de un tradeoff entre m\'as r\'apido y menor precisi\'on en la transici\'on (interpolaci\'on lineal), o m\'as lento y mayor precisici\'on (splines). 
