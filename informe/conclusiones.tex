Luego de terminar la experimentaci\'on computacional y el an\'alisis de artifacts y otros efectos de video, deber\'iamos elegir el mejor m\'etodo entre los propuestos para alentar los videos.

Nos podemos guiar con los siguientes datos:

\begin{itemize}
	\item El m\'etodo de los vecinos m\'as cercanos, a pesar de ser el m\'as r\'apido de los m\'etodos, tiene un resultado con un error considerablemente mayor que los otros dos. Adem\'as, se pierde la fluidez y la ilusi\'on de movimiento continuo del video, as\'i que puede ser tranquilamente descartado entre los m\'etodos a usar.
	\item Cuando alenta el video un poco, hasta 3 o 4 veces el largo del video original, los m\'etodos de interpolaci\'on lineal e interpolaci\'on por splines tienen similar error cuadr\'atico medio, aunque es menor en el primer m\'etodo y en casi todos los frames. Cuando se alenta todav\'ia m\'as, la interpolaci\'on lineal pasa a ser considerablemente mejor que el caso de los splines.
	\item El m\'etodo de los splines es mucho m\'as lento que los otros dos m\'etodos, y separar los frames en grupos peque\~nos y calcular un spline por cada grupo no ayuda mucho a la velocidad, a pesar de hacer que el error sea todav\'ia mayor.
	\item El m\'etodo de interpolaci\'on por splines genera una cantidad considerale m\'as de artifacts que el de intepolaci\'on lineal, en especial cuando se consideran saltos de un valor alto.
	\item Aunque el error medio es mayor, cuando se analiza a ojo se puede notar que el m\'etodo de los splines es bastante m\'as flu\'ido que el de interpolaci\'on lineal. Esto se debe a que el cerebro humano analiza las imagenes de una manera bastante diferente que una computadora, y es una ventaja ya lo ideal es que el resultado sea visto por personas.
\end{itemize}

A pesar de que el \'ultimo punto aventaja ligeramente al m\'etodo de splines sobre el de interpolaci\'on lineal, nosotros llegamos a la conclusi\'on de que esta ventaja no supera a la mejoras sobre el tiempo de corrida y sobre error cuando el video se hace mucho m\'as lento que el original del segundo m\'etodo. Por lo tanto, en el caso general preferi\'iamos usar el m\'etodo de \textbf{interpolaci\'on lineal} para alentar un video.
